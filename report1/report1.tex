%!TEX encoding = UTF-8 Unicode
% ================================================================================
\documentclass[
    fontsize=12pt,
    headings=small,
    parskip=half,           % Ersetzt manuelles Setzen von parskip/parindent.
    bibliography=totoc,
    numbers=noenddot,       % Entfernt den letzten Punkt der Kapitelnummern.
    open=any,               % Kapitel kann auf jeder Seite beginnen.
%   final                   % Entfernt alle todonotes und den Entwurfstempel.
    ]{scrreprt}
% ===================================Praeambel==================================
% Kodierung, Sprache, Patches {{{
\usepackage[T1]{fontenc}    % Ausgabekodierung; ermöglicht Akzente und Umlaute
                            %  sowie korrekte Silbentrennung.
\usepackage[utf8]{inputenc} % Erlaubt die direkte Eingabe spezieller Zeichen;
                            %  utf8 muss die Eingabekodierung des Editors sein.
\usepackage[ngerman]{babel} % Deutsche Sprachanpassungen (z.B. Überschriften).
\usepackage{microtype}      % Optimale Randausrichtung und Skalierung.
\usepackage[
    autostyle,
    ]{csquotes}             % Korrekte Anführungszeichen in der Literaturliste.
%\usepackage{fixltx2e}      % Patches fuer LaTeX2e - seit 2015 nicht mehr nötig
\usepackage{scrhack}        % Verhindert Warnungen mit älteren Paketen.
\usepackage[
  newcommands
]{ragged2e}                 % Verbesserte \ragged...Befehle
\PassOptionsToPackage{
  hyphens
}{url}                      % Sorgt für URL-Umbrüche in Fußzeilen u. Literatur
% }}}

% Schriftarten {{{
\usepackage{mathptmx}       % Times; modifies the default serif and math fonts
\usepackage[scaled=.92]{helvet}% modifies the sans serif font
\usepackage{courier}        % modifies the monospace font
% }}}

% Biblatex {{{
\usepackage[
    style=alphabetic,
    backend=biber,
    %backref=true
    ]{biblatex}             % Biblatex mit alphabetischem Style und biber.
\bibliography{\jobname.bib} % Dateiname der bib-Datei.
\DeclareFieldFormat*{title}{
    \mkbibemph{#1}}         % Make titles italics
% }}}

% Dokument- und Texteinstellungen {{{
\usepackage[
    a4paper,
    margin=2.54cm,
    marginparwidth=2.0cm,
    footskip=1.0cm
    ]{geometry}             % Ersetzt 'a4wide'.
\clubpenalty=10000          % Keine Einzelzeile am Beginn eines Absatzes
                            %  (Schusterjungen).
\widowpenalty=10000         % Keine Einzelzeile am Ende eines Absatzes
\displaywidowpenalty=10000  %  (Hurenkinder).
\usepackage{floatrow}       % Zentriert alle Floats
\usepackage{ifdraft}        % Ermöglicht \ifoptionfinal{true}{false}
\pagestyle{plain}           % keine Kopfzeilen
% \sloppy                    % großzügige Formatierungsweise
\deffootnote{1em}{1em}{
  \thefootnotemark.\ }      % Verbessert Layout mehrzeiliger Fußnoten

\makeatletter
\AtBeginDocument{%
    \hypersetup{%
        pdftitle = {\@title},
        pdfauthor  = \@author,
    }
}
\makeatother
% }}}

% Weitere Pakete {{{
\usepackage{graphicx}       % Einfügen von Graphiken.
\usepackage{tabu}           % Einfügen von Tabellen.
\usepackage{multirow}       % Tabellenzeilen zusammenfassen.
\usepackage{multicol}       % Tabellenspalten zusammenfassen.
\usepackage{booktabs}       % Schönere Tabellen (\toprule\midrule\bottomrule).
\usepackage[nocut]{thmbox}  % Theorembox bspw. für Angreifermodell.
\usepackage{amsmath}        % Erweiterte Handhabung mathematischer Formeln.
\usepackage{amssymb}        % Erweiterte mathematische Symbole.
\usepackage{rotating}
\usepackage[
    printonlyused
    ]{acronym}              % Abkürzungsverzeichnis
\usepackage[
    colorinlistoftodos,
    textsize=tiny,          % Notizen und TODOs - mit der todonotes.sty von
    \ifoptionfinal{disable}{}%  Benjamin Kellermann ist das Package "changebar"
    ]{todonotes}            %  bereits integriert.
\usepackage[
    breaklinks,
    hidelinks,
    pdfdisplaydoctitle,
    pdfpagemode = {UseOutlines},
    pdfpagelabels,
    ]{hyperref}             % Sprungmarken im PDF. Lädt das URL-Paket.
    \urlstyle{rm}           % Entfernt die Formattierung von URLs.
%\usepackage{breakurl}
%\def\UrlBreaks{\do\/\do-}
\usepackage{listings}       % Spezielle Umgebung für Quelltextformatierung.
    \lstset{                
        language=C,
        breaklines=true,
        breakatwhitespace=true,
        frame=l,            % Linie links: l, doppelt: L
		framerule=2.5pt,    % Dicke der Linie
		rulecolor=\color{gray},% Farbe der Linie
        captionpos=b,
        xleftmargin=6ex,
        tabsize=4,
        numbers=left,
        numberstyle=\ttfamily\footnotesize,
        basicstyle=\ttfamily\footnotesize,
        keywordstyle=\bfseries\color{green!50!black},
        commentstyle=\itshape\color{magenta!90!black},
        identifierstyle=\ttfamily,
        stringstyle=\color{orange!90!black},
        showstringspaces=false,
        }
%\usepackage{filecontents}  % Direktes Einfügen von Dateiinhalt. Wird hier für
                            %  die Verwendung einer .bib-Datei in dieser .tex-
                            %  Datei benötigt.
% }}}

% ===================================Dokument===================================

\title{Labreport 1}
\author{Enric Carstens, Matz Radloff}
% \date{01.01.2015} % Falls ein bestimmtes Datum eingesetzt werden soll, einfach
                    %  diese Zeile aktivieren.

\begin{document}

\begin{titlepage}
\begin{center}\Large
    % Universität Hamburg \par
    % Fachbereich Informatik
    \vfill
    \makeatletter
    {\Large\textsf{\textbf{\@title}}\par}
    \makeatother
    \bigskip
    \makeatletter
    {\@author} \par
    \makeatother
    \bigskip
    \makeatletter
    {\@date}
    \makeatother
    \vfill
    \vfill
\end{center}
\end{titlepage}

\chapter{Arbeiten mit der Linux-Kommandozeile (bash)}

\section{Hilfe zu Befehlen}

\subsection*{1.}
Durch den Befehl \emph{ls} lassen sich die Inhalte des Verzeichnisses aufrufen.

\subsection*{2.}
Ruft man den Befehl \emph{ls --help} auf, werden alle flags und dessen Funktionen angezeigt.

\subsection*{3.}
Mit dem Befehl \emph{script} wird ein typescript bestehend aus allen bisherigen Eingaben des Terminals erstellt. Dieser kann nützlich sein, um sich ausgeben zu lassen, welche Befehle man zur Bearbeitung der Aufgaben benutzt hat.


\section{Benutzerkonten und -Verwaltung}
\subsection*{0.}
Die folgenden Befehle sind mit \emph{sudo} auszuführen, da der normale User keinen Zugriff auf diese hat.

\subsection*{1.}
Einen neuen User mit dem Namen "labmate" kann man mit dem Befehl \emph{useradd -m labmate} erstellen. Das Passwort kann man dann mit Hilfe von \emph{passwd laborratte} setzen.

\subsection*{2.}
Die Gruppen eines Users kann man sich mit \emph{groups 'user'} anzeigen lassen.

\subsection*{3.}
Eine neue Gruppe lässt sich mit dem Befehl \emph{groupadd labortests} anlegen.

\subsection*{4.}
Den User kann man mit dem Befehl \emph{usermod -a -G labortests labmate} zu einer Gruppe hinzufügen. Der \emph{-a} Flag steht für "append" und \emph{-G} für Gruppe.

\subsection*{5.}
Die sudo-Berechtigung kann man einem User mit dem Befehl \emph{usermod -a -G sudo 'user'} zuweisen. Damit wird der User zu der sudo Gruppe hinzugefügt.

sudo visudo

\section{Datei- und Rechteverwaltung}

\subsection*{1.}
Ein neues Terminal lässt sich durch Klic

\subsection*{2.}
Mit dem Befehl \emph{su labmate} "switch user" wechselt man zu einem anderen Benutzer. Alternativ kann man \emph{su - labmate} verwenden, um eine neue Login-Shell zu starten, sodass die Umgebung inkl. aktuellem Pfad und Umgebungsvariablen zurückgesetzt wird.

\subsection*{3.}
Das Programm \emph{cd} "change directory" ändert den aktuellen Pfad. Mit \emph{cd /home/labmate} erreicht man das home-Verzeichnis des Users labmate.
\emph{pwd} "print working directory" gibt den aktuellen Pfad aus.

\subsection*{4.}
Mit dem Kommando \emph{mkdir} "make directory" lassen sich neue Verzeichnisse anlegen. \emph{mkdir labreports} erstellt das Verzeichnis "labreports".

\subsection*{5.}
\emph{cd labreports} wechselt in das Verzeichnis "labreports".

\subsection*{6.}
\emph{touch} ist ein Programm zum verändern von Datei-Zeitstempeln, allerdings erstellt es auch die angegebene Datei, falls diese noch nicht existiert. Entsprechend lässt sich die leere Datei mit \emph{touch bericht1.txt} anlegen.
\emph{nano} ist ein interaktiver Editor, der im Terminal läuft. Mit \emph{nano bericht1.txt} öffnet man die neue Datei und kann diese per normaler Tastatureingaben editieren. Mit \emph{Strg+x}, \emph{Y} und \emph{Enter} schließt man die Datei und besätigt, dass die Änderungen gespeichert werden sollen.

\subsection*{7.}
Damit nur Teilnehmer der Gruppe "labortests" Zugriff auf diese Datei haben, ändert man zunächst die Gruppe der Datei mit \emph{sudo chgrp labortests bericht1.txt} "change group". Dies muss mit "sudo" durchgeführt werden, da ein normaler User keine andere Gruppe setzen darf.
Mit \emph{ls -l bericht1.txt} überprüft man, dass der Eigentümer "labormate" ist und die rw-Rechte gesetzt sind, was standardmäßig beim Erstellen einer Datei der Fall ist. Die Gruppe besitzt zu diesem Zeitpunkt nur die Leseberechtigung "r". Schreibrechte für alle Teilnehmer der Gruppe "labortests" fügen wir mit dem Befehl \emph{chmod g+w bericht1.txt} hinzu. Das "g" steht hier für "group", "+" für das Hinzufügen und "w" für die Schreibrechte.

\subsection*{8.}
Ohne weitere Parameter läd das Programm \emph{wget [URL]} die Datei mit der angegebenen URL in das aktuelle Verzeichnis. Da wir uns bereits in dem neuen Verzeichnis befinden, führen wir \emph{wget http://www.uni-hamburg.de/index.html} aus.

\subsection*{9.}
Mit dem Befehl \emph{cd ~} wechseln wir in das home-Verzeichnis. "~" wird vom Terminal immer mit dem Pfad des home-Verzeichnis ersetzt. Per \emph{chmod 0660 labortests} setzen wir die Berechtigungen. Dies setzt allerdings nicht die "x"-Flag, welche für Verzeichnisse die "Search"- bzw. Durchsuchen-Berechtigung darstellt. Wenn nun versucht wird, auf das Verzeichnis zuzugreifen, z.B. mit \emph{ls -l labortests}, sieht man nur mehrere "?" statt der Informationen über die Berechtigungen und bei direktem Zugriff auf die neu erstellte Datei, z.B. mit \emph{cat labortests/bericht1.txt} wird dies mit "Permission denied" abgelehnt.
Eine bessere Lösung währe \emph{chmod 770 labortests}, welche auch die Durchsuchen-Berechtigung setzt.

\subsection*{10.}
Das Wechseln in das Verzeichnis "/root" ist nicht möglich, da es mit den Berechtigungen "750" allen nicht-root Benutzern das Durchsuchen verbietet.

\subsection*{11.}
\emph{sudo mkdir /opt/test} erstellt das Verzeichnis "test" in "/opt". \emph{cd /opt/test} wechselt dort hin.
Da wir der Ordner per "sudo" angelegt werden musste, da der Benutzer "labmate" keine Rechte hatte, um in "/opt" einen Ordner zu erstellen, ändern wir nun den Besitzer mit \emph{sudo chown labmate .}.
\emph{sudo usermod -aG labortests user} fügt den Benutzer "user" der Gruppe "labortests" hinzu, welche nun für den neuen Ordner verwendet wird: \emph{sudo chgrp labortests .} ändert die Gruppe des Ordners.
\emph{chmod 0770 .} änder die Rechte des Ordner so, dass der Besitzer und die Gruppe Lese-, Schreib- und Durchsuch-Rechte besitzen. Alle anderen Benutzer haben keinen Zugriff.

\subsection*{12.}
Mithilfe des Programm \emph{cp [SRC] [DEST]} lassen sich Dateien oder Ordner "[SRC]" in das Verzeichnis "[DEST]" kopieren. \emph{cp ~/labortests/index.html .} kopiert die heruntergeladene Datei in das aktuelle Verzeichnis.

\subsection*{13.}
Um die Rechte entsprechend zu ändern benötigt die Datei "640" Berechtigungen. "6" gibt mit 4+2 dem Besitzer Lese- und Schreibzugriff. "4" gibt der Gruppe, und nach Änderung der Gruppe der Datei auf "labortests" dem Benutzer "user", Leserechte. "0" stellt sicher, dass kein anderer Benutzer Zugriffsrechte besitzt. \emph{sudo chgrp labortests index.html} und \emph{chmod 640 index.html} erreichen dies.

\subsection*{14.}
Mit \emph{exit} schließen wir die aktuelle shell und wechseln zurück zum Benutzer "user".

\subsection*{15.}
Dies gelingt nicht, da eine neue Shell gestartet werden muss, um die Gruppen zu aktualisieren. Mit \emph{su - user} loggen wir uns erneut ein. Nun klappt der Zugriff auf die Datei mit \emph{cat /opt/test/index.html}.

\subsection*{16.}
Nano zeigt sofort eine Warnung an "File 'index.html' is unwritable". Entsprechend erhält man beim Speichern den Fehler "Error writing index.html: Permission denied".

\subsection*{17. + 18.}
Mit \emph{cd /opt/test}, \emph{cp index.html userindex.html} und \emph{nano userindex.html} erstellt man eine Kopie der Datei und öffnet sie in nano. Diese lässt sich nun speichern, da beim Kopieren eine neue Datei erstellt wurde und der Benutzer user dadurch automatisch zum Besitzer wurde.

\subsection*{19.}
Beim versuchten Löschen erhält man "rm: remove write-protected regular file 'index.html'?". Wenn man dies mit "y" bestätigt, wird die Datei gelöscht.


\section{Administration und Aktualisierung}

\subsection*{1.}
Durch den Befehl \emph{apt} wird ein command-line Interface aufgerufen. Ruft man stattdessen \emph{apt update} auf, werden alle packages von ihren jeweiligen Quellen neu synchronisiert. Verwendet man allerdings \emph{apt upgrade} kann man verfügbare upgrades aller installierten packages auf dem System installieren. 

\subsection*{2.}
Um das Programm 'cowsay' zu installieren, benutzen wir den Befehl \emph{apt install cowsay}.

\section{Prozesse und Prozessverwaltung}

\subsection*{1.}
Der Befehl \emph{ps} zeigt Information über den aktiven Prozess an, hierbei kann man mit Flags filtern, welche Prozesse bzw. von welchen Usern man sich diese ausgeben lassen möchte. Mit \emph{top} kann man sich alle Linux Prozesse dynamisch in real-time anzeigen lassen. Auch hier kann man durch viele verschiedene Flags die Ausgabe modifizieren.

\subsection*{2.}
Das Programm 'top' wird durch den Befehl \emph{top} gestartet.

\subsection*{3.}
Erst wechseln wir den User mit Hilfe von \emph{su labmate}. Anschließend öffnen wir ein weiteres Terminal.
Führt man darauf den Befehl \emph{cat /dev/urandom} aus, wird ein Stream von Bytes an das Terminal gesendet, die nicht alle angezeigt werden können. Dies bringt das Terminal zum Absturz. Im top Fenster sieht man, dass die Auslastung des Prozesses aufgrund des \emph{cat} Befehls schnell auf 100\% steigt.

\subsection*{4.}


\subsection*{5.}
Da der User 'labmate' über sudo Rechte verfügt, ist dieser Berechtigt das System durch den Befehl \emph{sudo reboot} neu zu starten. Sollte ein User keine sudo Rechte besitzen, kann er dadurch nicht das System neu starten.

\subsection*{6.}
Erst legen wir im Verzeichnis \emph{/home/labmate/} eine neue Datei mit dem Namen 'stempel.txt' an. Dies erfolgt durch den Befehl \emph{touch stempel.txt}. Anschließend muss man mit \emph{crontab -e} das Programm ausführen. Dort kann man dann mit dem Befehl \emph{*/5 * * * * date \textgreater\textgreater /home/labmate/stempel.txt} einen Cronjob anlegen, der alle 5 Minuten die Uhrzeit und Datum an die Datei 'stempel.txt' anhängt. Die ersten fünf Argumente stehen hierbei für die Frequenz der Ausführung des Jobs, beginnend mit den Minuten (in diesem Fall bedeutet */5 alle 5 Minuten), Stunden (Stern für jede Stunde), Tagen, Monaten und schließlich den Wochentagen. Der Befehl \emph{date} erzeugt eine Ausgabe mit der aktuellen Uhrzeit und Datum, die dann an die Datei 'stempel.txt' gesendet wird.



\section{VMware-Tools installieren}

\section{Bedienung von VMware}




\fullcite{Test}

\end{document}